\documentclass[10pt,uplatex]{jsarticle}
\usepackage[jis2004]{otf}
\usepackage[top=5truemm,bottom=20truemm,left=10truemm,right=10truemm]{geometry}
\begin{document}
\section*{救急医学レポート『敗血症の新しい定義について』}

\begin{flushright}
2017 年 2 月 13 日 02113072 井上 大輔
\end{flushright}

\section{敗血症の概要}

丸藤 [0, p. 153] によれば,菌とその代謝産物等が感染病巣から血中に侵入し,種々の臓器・組織に転移病巣を作り,中毒症状を呈す病態が敗血症であると,歴史的に定義されてきた.その後 1990 年前後に見直しが行われ,感染に起因する全身性の炎症反応が敗血症であると定義が変更された.

丸藤 [0, p. 153] によれば,全身性炎症反応症候群 ( SIRS, systemic inflammatory response syndrome ) のうち,生能侵襲が感染のものを敗血症と定義できる.

日本救急医学会 [1, pp.292 - 293] によれば,敗血症は,外傷・手術等の全身性炎症の 2 次性侵害刺激として SIRS を増悪させる.しかも,心不全・癌といった慢性疾患による免疫低下を基盤として合併して SIRS を誘導する.敗血症は蛋白異化・脂肪異化・痩せという形で生体のホメオスタシスを障害する.

Longo ら [2, p. 1926] によれば,局所の炎症によっても遠隔臓器の機能障害・低血圧が発生し得るので,微生物が血中に侵入していることは敗血症の発生に対し必須ではない.

\section{新しい定義 ( 2016 年 )}

1991 年・ 2001 年の敗血症の定義が用いられた時代に,敗血症の新定義・診断基準を策定する必要があるという主張があった.この目的のため,米国集中治療医学会と欧州集中治療医学会の専門家 19 人から成るタスクフォースが設置された.このタスクフォースは,会議や,データベースを用いた新たな解析研究や,投票等を通じてドラフトを作成した.関連専門家団体による査読の機会も,広く,国際的に設けられた.最終的に,日本集中治療医学会を含む 31 の専門家団体の賛同を得,確定版の公表に至った. 2016 年 2 月 22 日,第45回米国集中治療医学会において,「敗血症および敗血症性ショックの国際コンセンサス定義第3版 ( Sepsis-3 ) 」が報告され,同時に「 The Journal of the American Medical Association 」に掲載された.それが含むのは,「敗血症」「敗血症性ショック」の定義と診断基準である.

山本ら [3] に依拠して内容を以下に記す.

\subsection{敗血症の定義}

感染に対する宿主生体反応の調節不全で,生命を脅かす臓器障害

\subsection{敗血症の診断基準}

感染症が疑われ, SOFA スコアが 2 点以上増加したもの

合併症のない患者であれば, 0 点がベースラインである.合併症があれば,その時点での SOFA スコアがベースラインである.

ただし, ICU の外では qSOFA スコアが 2 点以上であれば,敗血症を疑い臓器障害を評価することを推奨する.



\subsection{敗血症性ショックの定義}

敗血症の部分集合であり,実質的に死亡率を上昇させる重度の循環・細胞・代謝の異常を呈するもの

\subsection{敗血症性ショックの診断基準}

十分な輸液負荷にもかかわらず,平均動脈圧 66 mmHg 以上を維持するために血管作動薬を必要とし,かつ血清乳酸値が 2 mmol / L を超えるもの

\section{SOFA スコアの定義}

上の敗血症の診断基準は SOFA スコアを参照している.だから,これも記す.山本ら [3] によれば,下表の通りである:

\begin{table}[htb]

\begin{tabular}{|p{3.5cm}|p{2.5cm}|p{2.5cm}|p{2.5cm}|p{2.5cm}|p{2.5cm}|}  \hline
 & 0 点 & 1 点 & 2 点 & 3 点 & 4 点 \\  \hline
呼吸器 ( $\mathrm{PaO}_2 / \mathrm{F_IO_2}$ ) & $\geq$ 400 & $<$ 400 & $<$ 300 & $<$ 200 + 呼吸補助 & $<$ 100 + 呼吸補助 \\  \hline
凝固能 ( 血小板数 [ $\times 10^3 / \mu \mathrm{L} ])$ & $\geq$ 150 & $<$ 150 & $<$ 100 & $<$ 50 & $<$ 20 \\  \hline
肝臓 ( ビリルビン [ mg / dL ] ) & $<$ 1.2 & 1.2 - 1.9 & 2.0 - 5.9 & 6.0 - 11.9 & $>$ 12 \\  \hline
循環器 & 
 MAP $\geq$ 70 mmHg & MAP $<$ 70 mmHg & 
 DOA $<$ 5 $\gamma$ あるいは DOB 使用 &
 DOA 5.1 - 15 $\gamma$ あるいは Ad $\leq$ 0.1 $\gamma$ あるいは NOA $\leq$ 0.1 $\gamma$ &
 DOA $>$ 15 $\gamma$ あるいは Ad $>$ 0.1 $\gamma$ あるいは NOA $>$ 0.1 $\gamma$ \\  \hline
中枢神経系 ( GCS ) & 15 & 13 - 14 & 10 - 12 & 6 - 9 & $<$ 6 \\  \hline
腎 ( クレアチニン [ mg / dL ] ,尿量 [ mL / day ] ) & $<$ 1.2 & 1.2 - 1.9 & 2.0 - 3.4 & 3.5 - 4.9, $<$ 500 & $>$ 5.0, $<$ 200 \\  \hline
\end{tabular}
\end{table}

ただし,

\begin{itemize}
\item DOA:ドーパミン
\item DOB:ドブタミン
\item Ad:アドレナリン
\item NOA:ノルアドレナリン
\end{itemize}

\section{qSOFA スコアの定義}

上の敗血症の診断基準は qSOFA スコアを参照している.だから,これも記す. Seymour ら [4] によれば,下表の通りである:

\begin{table}[htb]
\begin{center}
\begin{tabular}{|p{5cm}|p{2.5cm}|p{2.5cm}|p{2.5cm}|p{2.5cm}|p{2.5cm}|}  \hline
 & 0 点 & 1 点 \\  \hline
循環器 ( 収縮期血圧 [ mmHg ] )& $>$ 100 & $\leq$ 100 \\ \hline
呼吸器 ( 呼吸数 [ / min ] )& $<$ 22 & $\geq$ 22 \\ \hline
中枢神経系 ( GCS ) & $\geq$ 15 & $<$ 15 \\ \hline
\end{tabular}
\end{center}
\end{table}

\section{かつての定義}

\subsection{1991 年の定義}

日本救急医学会 [5, pp. 642 - 643] によれば,1991 年に,アメリカ胸部疾患学会と集中治療学会の合同コンセンサス委員会は,敗血症・重症敗血症を定義した.それを以下記す.

\subsubsection{敗血症の定義}

感染に対する全身性反応.感染の結果として以下の 2 つないしそれ以上の項目により明らかにされる.

\begin{itemize}
\item 体温 $>$ 38 ℃ ないし $<$ 36 ℃
\item 心拍数 $>$ 90 / min
\item 呼吸数 $>$ 20 / min ないし PaCO$_2$ $<$ 32 mmHg
\item 白血球数 $>$ 12000 / $\mu{}$L , $<$ 4000 / $\mathrm{\mu}$L ないし桿状核好中球 $>$ 10 \%
\end{itemize}

\subsubsection{重症敗血症の定義}

臓器機能障害,低灌流,あるいは低血圧を伴う敗血症.低還流 ( 原文ママ ) と還流 ( 原文ママ ) 異常は乳酸アシドーシス,乏尿,ないし精神状態の急性変化を含むが,これだけに限定されない

\subsection{2001 年の定義}

山本ら [3] によれば,2001 年,欧州・米国の多様な専門家が,敗血症の新たな定義を提案した.

\subsubsection{敗血症の定義}

感染に起因する全身症状を伴った症候

\subsubsection{敗血症の診断基準}

感染症の存在が確定もしくは疑いであり,かつ下記のいくつかを満たす.

\vspace{1em}
\textgt{全身所見}

\begin{itemize}
\item 体温 $>$ 38.3 ℃, $<$ 36 ℃
\item 心拍数 $>$ 90 / 分,もしくは $>$ ( 年齢平均 ) $ + 2 \mathrm{SD} $
\item 頻呼吸
\item 精神状態の変容
\item 著名な浮腫または体液バランス過剰 ( 24 時間以内で 20 mL / kg 以上 )
\item 高血糖:糖尿病の既往がない症例で血糖値 $>$ 120 mg / dL
\end{itemize}


\textgt{炎症所見}
\begin{itemize}
\item WBC $>$ 12000 / $\mu$L, $<$ 4000 / $\mu$L
\item 白血球数正常で幼若白血球 $>$ 10 \%
\item CRP $>$ ( 基準値 ) $+ 2\mathrm{SD}$
\item プロカルシトニン $>$ ( 基準値 ) $+ 2\mathrm{SD}$
\end{itemize}

\textgt{循環所見}
\begin{itemize}
\item 血圧低下: SBP $<$ 90 mmHg , MAP $<$ 70 mmHg ,もしくは成人で正常値より 40 mmHg を超える低下,小児で正常値より $2\mathrm{SD}$ を超える低下
\item 混合静脈血酸素飽和度 $<$ 70 \%
\item 心係数 $>$ 3.5 L / 分 / $\mathrm{m}^2$
\end{itemize}

\textgt{臓器障害所見}
\begin{itemize}
\item 低酸素血症: $\mathrm{PaO}_2 / \mathrm{F_IO_2} < 300$
\item 急性乏尿:尿量 $<$ 0.5 mL / kg / 時 が少なくとも 2 時間持続
\item クレアチニンの増加: $>$ 0.5 mg / dL
\item 凝固異常: PT-INR $>$ 1.5 ,もしくは APTT $>$ 60 秒
\item イレウス:腸蠕動音の消失
\item 血小板減少 $<$ 100000 / $\mu$L
\item 高ビリルビン血症 $>$ 4 mg / dL
\end{itemize}

\textgt{炎症所見}
\begin{itemize}
\item 高乳酸血症 $>$ 3 mmol / L
\item 毛細血管再灌流減少,もしくは mottled skin
\end{itemize}

\subsubsection{重症敗血症の定義}

臓器障害を伴う敗血症

\subsubsection{重症敗血症の診断基準}

臓器障害は以下のいずれかを満たす.

\begin{itemize}
\item 敗血症に起因する低血圧
\item 乳酸高値
\item 2 時間以上の適切な輸液蘇生を行っても尿量が 0.5 mL / kg / 時未満
\item 感染巣が肺炎でなく $\mathrm{PaO}_2 / \mathrm{F_IO_2} < 250$ の急性肺傷害
\item 感染巣が肺炎であり $\mathrm{PaO}_2 / \mathrm{F_IO_2} < 200$ の急性肺傷害
\item クレアチニン $>$ 2 mg / dL
\item ビリルビン $>$ 2 mg / dL
\item 血小板数 $<$ 10000 / $\mu$L
\item 凝固障害 ( PT-INR $>$ 1.5 )
\end{itemize}

ただし,「敗血症に起因する低血圧」は,

\begin{itemize}
\item 収縮期血圧 $<$ 90 mmHg または
\item 平均動脈圧 $<$ 70 mmHg または
\item 収縮期血圧の 40 mmHg を超える低下 または
\item 他の低血圧要因がなく,その年齢の血圧より $2\mathrm{SD}$ 以上の低下
\end{itemize}

と定義する.

\subsubsection{敗血症性ショックの定義}

十分な輸液負荷にもかかわらず持続する低血圧を伴う敗血症

\subsection{2001 年の定義の問題点}

2001 年の定義には問題点が多くあった.

第 1 に, 1991 年のものに比べ,診断基準の項目数が多かった.

第 2 に,診断基準をいくつ満たせば診断できるのかの明確な記載がなかった.

第 3 に,「敗血症」「重症敗血症」の定義のカットオフ値には科学的根拠がなかった.「敗血症性ショック」の定義は血管作動薬の有無や血圧のカットオフ値を含まなかったため,臨床研究においてはそれぞれの研究によって,異なるカットオフ値を採用していた.

第 4 に,敗血症性ショックの定義が,病態を正確には反映していなかった.定義は,血圧のみを指標としているが,敗血症性ショックの病態は細胞・代謝の異常を伴うので,この定義では不十分である.

これらが定義自体の問題点である.このうえ,実用上も,敗血症の診断に対する感度・特異度が, 1991 年の定義と大差なかった.

これらが原因となり, 2001 年の定義の発表の後も,臨床現場・臨床研究においては 1991 年の定義が引き続き使われることが多かった.

\subsection{1991 年の定義と 2001 年の定義の共通の問題点}

1991 年の定義と 2001 年の定義はともに,「敗血症」と「重症敗血症」を別の術語として定義している.しかし,臨床現場ではこれらを混同して使うことが多かった.定義で言う「重症敗血症」の意味で「敗血症」という語を使うことも多かった.

\section{2016 年の定義の企図}

\subsection{「重症敗血症」の削除}

2016 年の定義は,「重症敗血症」を削除している. 1991 年・ 2001 年の定義で「敗血症」「重症敗血症」に分類していた概念を「敗血症」に一本化している.旧来の定義のこの二つの術語を混同する用法が多かったためである.

\subsection{SOFA スコアの採用}

2016 年の定義は, SOFA スコアを用いて敗血症を定義している.

SOFA スコアは, 1994 年に Vincent らが,敗血症での臓器障害のスコアリングを目的に作成したものである.このときは, SOFA は Sepsis-related Organ Failure Assessment の略だった.この後,敗血症以外の集中治療患者にも用いるようになった.これに伴い, SOFA は Sequential Organ Failure Assessment の略であると変更された.

Seymourらは,大規模なデータベースを使って,

\begin{itemize}
\item SOFA
\item qSOFA
\item LODS ( Logistic Organ Dysfunction System )
\item SIRS
\end{itemize}

のそれぞれと院内死亡率の関連を調べた.この結果,

\begin{itemize}
\item SOFA スコアと LODS は, SIRS と qSOFA よりも,死亡リスクとの相関が高い.
\item SOFA スコアと LODS は,死亡リスクとの相関がほぼ等しい.
\end{itemize}

と判明した.さらに, LODS が SOFA スコアより複雑であることも考慮して,「死亡リスクの評価のための臓器障害評価には SOFA が最適である」と結論した.

SOFA スコアが 2 点以上増加すると,院内死亡率が約 10 \%増加することが既にわかっている.これを踏まえて, 2016 年の敗血症の定義では,「感染症が疑われ, SOFA スコアが 2 点以上増加したもの」が診断基準となっている.

\subsection{qSOFA スコアの採用}

SOFA スコアの測定には,動脈血液ガスを含めた採血検査が必要である.このため, ICU の外で敗血症を迅速に診断するには, SOFA スコアは不便である.この弱点を克服するため, qSOFA ( quick SOFA ) スコアが作成された.院外・救急・一般病棟の患者に対して qSOFA スコアを用いることで,敗血症を簡便に発見できると期待されている.

\subsection{敗血症性ショックの定義への「細胞・代謝」の異常の追加}

1991 年・ 2001 年の「敗血症性ショック」の定義が病態を正確には反映していなかったことを先述した.この短所を克服するために, 2016 年の「敗血症性ショック」の定義は「敗血症の部分集合であり,実質的に死亡率を上昇させる重度の循環・細胞・代謝の異常を呈するもの」となっている.

\subsection{敗血症性ショックの診断基準の明確化}

1991 年・ 2001 年の「敗血症性ショック」の診断基準は,客観的な指標を欠いていた.特に,「十分な輸液負荷」「低血圧」の定義が明確でなかった.「敗血症性ショック」を客観的な指標を用いて定義するために,タスクフォースが主導して Delphi 法によって合意を形成した.

この結果,血圧については,平均動脈圧 65 mmHg をカットオフ値とするよう合意に至った.他方,「十分な輸液」「血管作動薬の必要性」については,合意に至らなかった.これらの指標は,治療を担当する医師の主観に強く依存し,さらに,モニタリング方法・鎮静等他のマネジメントとも相互に影響するため,客観的な指標を定めることが困難であったと考えられる.

2016 年の定義では,敗血症性ショックの定義は「実質的に死亡率を上昇させる重度な循環・細胞・代謝の異常」である.タスクフォースは,診断基準を変更するに際し,診断基準も新しい定義を反映するよう考慮した.こちらも,タスクフォースの主導により, Delphi 法により,

\begin{itemize}
\item 十分な輸液負荷にもかかわらず平均動脈圧が 65 mmHg 未満である.
\item 血清乳酸値が 2 mmol / L より大きい.
\item 血管作動薬を使用した.
\end{itemize}

の 3 項目が診断基準の候補として適当であるという合意に至った.

旧定義の敗血症患者,すなわち SIRS 2 項目かつ 1 つ以上の臓器障害を持つ患者を,上記 3 項目の組み合わせで 6 群に分け,院内死亡率を調べた.結果は下表の通りである.

\begin{table}[htb]
\begin{center}
\begin{tabular}{|p{1cm}|p{3cm}|p{3cm}|p{3cm}|p{3cm}|}  \hline

群 & 十分な輸液負荷にもかかわらず平均動脈圧 $<$ 65 mmHg & 血清乳酸値 $\geq$ 2 mmol / L & 血管作動薬の使用 & 院内死亡率 \\ \hline
1 & T & T & T & 42.3 \% \\ \hline
2 & T & F & T & 30.1 \% \\ \hline
3 & T & T & F & 28.7 \% \\ \hline
4 & F & T & F & 25.7 \% \\ \hline
5 & F ( 輸液負荷なし ) & T & F & 29.7 \% \\ \hline
6 & T & F & F & 18.7 \% \\ \hline

\end{tabular}
\end{center}
\end{table}


表の通り,

\begin{itemize}
\item 十分な輸液負荷にもかかわらず平均動脈圧が 65 mmHg 未満である.
\item 血清乳酸値が 2 mmol / L より大きい.
\item 血管作動薬を使用した.
\end{itemize}

の 3 項目全てを見たす群の死亡率は 42.3 \% だった.かつ,他群と有意差があった.この結果,これら 3 項目を全て用いて,敗血症性ショックの診断基準は「十分な輸液負荷にもかかわらず,平均動脈圧 66 mmHg 以上を維持するために血管作動薬を必要とし,かつ血清乳酸値が 2 mmol / L を超えるもの」となった.

\section{残された課題}

\subsection{SOFA スコアについて}

SOFA スコアでは,意識評価の指標として, GCS を用いている. GCS では,鎮静・挿管患者や全身麻酔患者において,スコアリングに統一ルールがない.例として,意識清明であるが処置のため鎮静・挿管が必要である患者を考える.この患者は,実質的には意識障害がない.このような場合,

\begin{itemize}
\item 鎮静前の意識清明な状態
\item 現時点で推測される意識状態
\item 鎮静・挿管状態
\end{itemize}

のいずれでスコアをつけるかによって,スコアが大きく変動する. GCS はこれらのうちどれを採用するかを定めていない.よって, GCS を敗血症の診断に用いるならば,これらのうちどれを採用するか,統一基準を追加する必要がある.

SOFA スコアの循環評価の項目では,平均動脈圧 70 mmHg以下をカットオフ値にしている.他方,敗血症性ショックの診断基準では平均動脈圧 65 mmHg以下を採用している.これらは統一性を欠いている.

\subsection{qSOFA スコアについて}

qSOFA スコアが2点以上増加した場合, ICU 長期滞在や院内死亡率上昇と関連することが証明されている.しかし,敗血症のスクリーニングとしての有用性は未だ確立していない.他のショック,せん妄,頻呼吸を引き起こす疾患においても qSOFA スコアは陽性となり得る. qSOFA スコアは,敗血症に対し特異度が低い可能性がある.

過去には, SIRS による過剰診断が指摘された.これと同様, qSOFA スコアは取り扱いに注意が必要である。今後, qSOFA スコアを敗血症のスクリーニングとして用いてゆくには,感度・特異度を検証しなければならない. 

\section*{参考文献}

[0] 丸藤哲 2003:臨床研修救急一直線 ( 南江堂 )

[1] 日本救急医学会 ( 監修 ) 2014 :標準救急医学 ( 医学書院 )

[2] Dan L. Longo ら 2012 :ハリソン内科学 ( メディカル・サイエンス・インターナショナル )

[3] 山本良平ら 2016 :敗血症の新定義・診断基準を読み解く ( http://www.igaku-shoin.co.jp/paperDetail.do?id=PA03169\_01 , 2017 年 2 月 12 日閲覧 )

[4] Christopher W. Seymour ら: What is qSOFA? ( http://www.qsofa.org/what.php , 2017 年 2 月 13 日閲覧 )

[5] 一般社団法人 日本救急医学会 ( 監修 ) 2011:救急診療指針 ( へるす出版 )

\end{document}
