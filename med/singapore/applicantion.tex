\documentclass[10pt,uplatex]{jsarticle}
\usepackage[jis2004]{otf}
\usepackage[top=10truemm,bottom=20truemm,left=10truemm,right=10truemm]{geometry}
\begin{document}
\section*{志望理由書}

\begin{flushright}
2017 年 4 月 14 日 02113072 井上 大輔
\end{flushright}

私がこの派遣を志望するにあたり,関心を持つ分野とその理由及び抱負を述べる.

\section{報道の自由の制限された国を見る.}

WHO ( World Health Organization ) [0] は「 health 」を「 Health is a state of complete physical, mental and social well-being and not merely the absence of disease or infirmity. 」と定義している.健康の条件に社会的環境を含めていることに私は注目している.社会的環境について言えば,現代において我々人類は自分のあり方を自分で決めること ( いわゆる自己統治・自己決定 ) を普遍の価値として承認しており,そのために民主的な政府を樹立することを普遍かつ天賦の権利として認めている.これを擁護することは医師の責務の一つであると私は信じている.

憲法学においては,表現の自由が民主主義 process において不可欠の権利であるという説が支配的である ( 松井 [1] ).表現のなかでも報道の自由は民主的な政府を維持するために特に重要である.

国境なき記者団 [2] の「報道の自由度ランキング」 2017 年版で,日本は 72 位, Singapore は 151 位だった.この ranking において日本は 2010 年に 11 位だったが,以降順位を下げ続けている.

加えて,日本の国会は過日,「組織的な犯罪の処罰及び犯罪収益の規制等に関する法律等の一部を改正する法律案」 ( いわゆる共謀罪法案 ) を可決した.共謀罪については, journalist ら有志が「言論の自由,表現の自由,報道の自由を破壊する」という内容の声明を発表した ( 朝日新聞[3] ).

この ranking については,問題を指摘する者もある.佐藤 [4] は,この ranking が専門家への聞き取りを根拠としており,専門家の主観に左右され,さらに一般国民の感覚を反映していない,と主張する.本田 [5] は,この ranking では Asia 諸国がおしなべて低く評価されており,地域ごとの偏りが大きい,また,米国 NGO の Freedom House の発表する「 Freedom of the Press 」では日本の score は悪くないと主張する.

共謀罪についても,報道の自由を損うものではないという主張がある.政府は,一般人は共謀罪の捜査対象とならないと主張している ( 朝日新聞[6] ).

今後我が国が報道の自由を失うかどうかは未知である.しかし,国を想う者は失った場合も想定して未来を考えるべきだとと私は信じている.

先述の通り, Singapore は報道の自由の制限された国である.理念上,報道の自由のない国は民主主義でない国であり,その国民は重大な人権侵害を受けている,ということになるが,現実を見れば必ずしもそうとは言えない事象に多数出会うことになるはずである.私は,この国の現在の姿を見ることにより,将来の日本の姿を見通したい.そして,日本において健康を増進するために医師として何をすべきか,より良い思想を持ちたい.

\section{IT の活用を見る.}

Singapore は IT ( information technology ) の先進国である. World Economic Forum [7] 発表の 2016 年「Networked Readiness Index」で, Singapore は 1 位,日本は 10 位だった.

厚生労働省が発表した 2014 年度の日本の国民医療費は 40兆8071億円だった.これは GDP の 8.33 \% に相当した.国民医療費はほぼ単調に増加している.さらに,日本の人口も減り続けている ( 国勢調査[8] ) .今後の日本は少ない金銭と人員で業務を遂行することになる.何らかの対策を打たなければならない.医療においても業務の効率化が必要である.

IT は効率化の強力な道具である.しかし今日の日本の医療は IT をうまく使っているとは言い難い.具体的に何が問題か,数例述べる.

まず,紙での情報伝達が多い.これは医療者と患者の間でもそうだし,医療者同士の間でもそうである.このせいもあってか,患者の既往歴や現在の藥の処方等の重要な情報を共有できていない例も散見される.

次に,情報への access の権限の設定が包括的である.現在普及しているある電子診療記録 system では,情報を PC の外の媒体 ( USB memory 等 ) に書き出すことが全くできない.このため,学会や研究会で症例を発表するとき,診療記録を PC で見ながら紙に写しその後自分の PC でその情報を使って slide を作ることを強いられる.紙に写したものを店や電車に置き忘れたら患者の個人情報の流出となる. Access 権限がもっと個別的に設定でき,適切な目的 ( 学会等 ) で適切な閲覧者 ( 学会参加者等 ) に適切な発表者が利用できるようになっていればこの害は避けられる.

最後に,病理診断・放射線診断に人工知能を利用できる可能性があると長年指摘されている.近い将来, computer や programming の知識は医師においても必須となるだろう.しかし,現在,医学生に対するこれらの教育は十分ではない.北海道大学医学科では教養課程で「情報学」の授業があるが,專門課程では情報活用の授業はない.教養課程の「情報学」は PC での文書作成,表計算, mail の送受信等,どの専門・どの職業でも使える技能を教えるものであり,医師の業務に特化したものではない.

他にも,私が気づいていない問題もあるだろう.私は Singapore に行って,いかに IT を医療に活用し効率を達成するかを学びたい.

\section*{参考文献}

[0] WHO: About WHO ( http://www.who.int/about/mission/en/ )

[1] 松井茂記: 日本国憲法, p. 446

[2] Reporters Without Borders: World Press Freedom Index 2017 年版 ( https://rsf.org/en/ranking )

[3] 朝日新聞 ( http://www.asahi.com/articles/ASK4W546TK4WUTIL03Q.html )

[4] 佐藤卓己  ( http://www.newsweekjapan.jp/stories/world/2017/02/post-7031.php )

[5] 本田康博  ( http://ironna.jp/article/3372 )

[6] 朝日新聞 (  http://www.asahi.com/articles/ASK584GTZK58UTFK009.html )

[7] World Economic Forum: Networked Readiness Index ( http://reports.weforum.org/global-information-technology-report-2016/networked-readiness-index/ )

[8] 国勢調査 ( http://www.stat.go.jp/data/kokusei/2015/ )
\end{document}
